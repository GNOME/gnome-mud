\documentclass{howto}
\usepackage{ltxmarkup}

\release{0.8.99}
\setshortversion{0.8}
\date{November 16, 2001}

\title{GNOME-Mud Python API}

\author{Petter E. Stokke}
\authoraddress{
	E-mail: \email{gibreel@project23.no}
}


\begin{document}

\maketitle

\begin{abstract}
\noindent
Documentation for the Python scripting API of the GNOME-Mud application.
\end{abstract}

\tableofcontents


\section{Introduction}
GNOME-Mud provides an embedded Python interpreter allowing advanced scripting. This document briefly
describes the Python API provided. It is assumed that the reader is familiar with basic Python programming; tutorials and additional
documentation are available on the \citetitle[http://www.python.org/]{Python Web page}.

\section{Basics}
When GNOME-Mud launches, it scans the \file{\$HOME/.gnome-mud} directory for Python scripts, ie. files ending with the suffix \file{.py}, and
loads each script found in turn. Each script should import the \refmodule{GnomeMud} module, which provides all the API functions necessary
for writing useful scripts.

\section{\module{GnomeMud} module}

\declaremodule{extension}{GnomeMud}
\modulesynopsis{Basic GNOME-Mud API}
\moduleauthor{Petter E. Stokke}{gibreel@project23.no}

The \module{GnomeMud} module provides access to the GNOME-Mud scripting API. Normally, a script would
call \function{register\_input\_handler()} and \function{register\_output\_handler()} as needed to register
callback functions, possibly adding some widgets using \function{add\_user\_widget()} if using \module{PyGTK},
and then have no further use of this module.

\begin{funcdesc}{add\_user\_widget}{widget, expand=TRUE, fill=TRUE, padding=5}
Adds a GTK+ widget, created by \module{PyGTK}, to the GNOME-Mud application window. The parameters are passed
more or less directly to \cfunction{gtk\_box\_pack\_start()}, and the meaning of \var{expand}, \var{fill} and \var{padding}
is to be found in the GTK+ documentation.

You would commonly use this function to build status displays or action buttons.

\strong{Note:} The \var{widget} parameter must refer to a genuine \module{PyGTK} widget. The function will only check
to ensure it is passed a Python object, and then blithely go on to assume it is the correct type of Python object.
The effect of passing a different kind of object to the function is undefined, although GTK+ can generally be trusted
to spot the error. However, Python will be completely unaware of the incident and assume everything proceeded in a
correct manner.

This function is only available if GNOME-Mud has been compiled with \module{PyGTK} support.
\end{funcdesc}

\begin{funcdesc}{connection}{}
Returns a \class{Connection} object for the currently active connection, meaning the connection tab
being currently displayed in the GNOME-Mud application window. This is useful for printing some
descriptive text during script initialisation, or displaying some especially urgent text to the user
from a callback function.
\end{funcdesc}

\begin{funcdesc}{register\_input\_handler}{function}
Registers a callback function for input received from a MUD connection. The callback function is passed two
parameters: a \class{Connection} object referring to the connection on which the input was received, and
a string containing the actual input. The return value of the function may be a string, which is then
further processed by GNOME-Mud as if it were the actual data received from the connection, or any other
type of Python object, in which case GNOME-Mud is passed the original string. Thus, you are able to modify
the data received on a socket before GNOME-Mud at large sees it, allowing you to gag input or perform text
substitutions.
\end{funcdesc}

\begin{funcdesc}{register\_output\_handler}{function}
Registers a callback function for data being sent to a MUD connection.{\nobreakspace}This works similarly to the callback
functions for \function{register\_input\_handler()} above, but applies to text entered by the user, or caused by
triggers or key bindings. It does \emph{not} apply to data sent by Python scripts using the \method{send()} method
provided by the \class{Connection} class.
\end{funcdesc}

\begin{funcdesc}{version}{}
This function simply returns the GNOME-Mud version that is running, as a string.
\end{funcdesc}

\subsection{\class{Connection} class}
\class{Connection} is an interface class for a GNOME-Mud connection, meaning a socket connection to a
MUD and its corresponding text display. It is normally used for sending data to the socket through the
\method{send} method and printing to the display using the \method{write()} method.

\begin{classdesc}{Connection}{}
The \class{Connection} class has no constructor; instances of this class are created by GNOME-Mud and
passed to the callback functions as needed, or accessed through the \refmodule{GnomeMud} module's
\function{connection()} function.

\begin{excdesc}{ConnectionError}
This exception is raised by \member{send} if a script is trying to write to a \class{Connection} class
that is not actually connected to a socket.
\end{excdesc}

\begin{methoddesc}{send}{data, echo=1}
Sends the string \var{data} to the connection, echoing it to the display if \var{echo} is true and the
user has requested echoing in the GNOME-Mud configuration.

\exception{ConnectionError} will be raised if this method is called on a \class{Connection} object that
is not actually connected to a socket (as determined by the \member{connected} member).
\end{methoddesc}

\begin{methoddesc}{write}{data}
Writes the string \var{data} to the display. It may contain ANSI formatting codes.
\end{methoddesc}

\begin{memberdesc}{connected}
An integer which is true if the \class{Connection} instance is connected to a socket, and false otherwise.
An attempt to \method{send()} data to an unconnected \class{Connection} will result in the \exception{ConnectionError}
exception being raised.
\end{memberdesc}

\begin{memberdesc}{host}
A string containing the host address of the remote end of the connection.
\end{memberdesc}

\begin{memberdesc}{port}
A string containing the TCP port of the remote end of the connection.
\end{memberdesc}

\begin{memberdesc}{profile}
A string containing the name of the GNOME-Mud profile that is active for this connection.
\end{memberdesc}

\end{classdesc}

\section{Example GNOME-Mud script}
Below is a simple example script that will replace occurrences of the text ``foo'' with ``bar'' in both
MUD input and user commands, as well as printing the obligatory text to the display upon initialisation to identify
its presence.

\begin{verbatim}
import GnomeMud

def replace_foo(conn, data):
    return data.replace("foo","bar")

conn = GnomeMud.connection()
conn.write("Hello, world!\n")
GnomeMud.register_input_handler(replace_foo)
GnomeMud.register_output_handler(replace_foo)
\end{verbatim}

\end{document}
